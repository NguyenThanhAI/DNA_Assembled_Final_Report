\documentclass[14pt, a4paper]{article}
\usepackage{minitoc}
\usepackage[left=3.00cm, right=2.5cm, top=2.00cm, bottom=2.00cm]{geometry}
\usepackage{amsmath}
\usepackage{amssymb}
\usepackage{amsthm}
\usepackage{thmtools}
\usepackage{mathtools}
\usepackage{graphicx}
%\usepackage{algpseudocode}
%\usepackage{algorithm}
\usepackage[ruled,vlined,linesnumbered,algosection]{algorithm2e}
\usepackage{blindtext}
\usepackage{setspace}
\usepackage[utf8]{inputenc}
\usepackage[utf8]{vietnam}
\usepackage[center]{caption}
\usepackage[shortlabels]{enumitem}
\usepackage{fancyhdr} % header, footer
\usepackage{hyperref} % loại bỏ border với mục lục và công thức
\usepackage[nonumberlist, nopostdot, nogroupskip]{glossaries}
\usepackage{glossary-superragged}
\usepackage{tikz,tkz-tab}
\setglossarystyle{superraggedheaderborder}
\pagestyle{fancy}
%\usepackage[style=numeric,sortcites]{biblatex}
%\addbibresource{ref.bib}
%\usepackage[numbers]{natbib}
\usepackage{indentfirst}
\usepackage{multirow}
\usepackage[natbib,backend=biber,style=ieee, sorting=ynt]{biblatex}
\bibliography{ref.bib}

\graphicspath{{./figures/}}


\makenoidxglossaries

% Danh mục thuật ngữ

\hypersetup{
    colorlinks=false,
    pdfborder={0 0 0},
}


\fancyhf{}
\rhead{\textbf{Môn học: Toán rời rạc và thuật toán}}
\lhead{\textbf{GVHD: PGS. TS. Nguyễn Thị Hồng Minh}}
\rfoot{\thepage}
\lfoot{\textbf{Nhóm học viên thực hiện: Nhóm 1}}
\renewcommand{\headrulewidth}{0.4pt}
\renewcommand{\footrulewidth}{0.4pt}


\numberwithin{equation}{section}
\numberwithin{figure}{section}

\setlength{\parindent}{0.5cm}

\setcounter{secnumdepth}{3} % Cho phép subsubsection trong report
\setcounter{tocdepth}{3} % Chèn subsubsection vào bảng mục lục

\newtheorem{dl}{Định lý}
\newtheorem{md}{Mệnh đề}
\newtheorem{bd}{Bổ đề}
\newtheorem{dn}{Định nghĩa}
\newtheorem{hq}{Hệ quả}

\numberwithin{dl}{section}
\numberwithin{md}{section}
\numberwithin{bd}{section}
\numberwithin{dn}{section}
\numberwithin{hq}{section}

\onehalfspacing
\AtBeginEnvironment{tabular}{\onehalfspacing}


\begin{document}
    \begin{titlepage}

        \newcommand{\HRule}{\rule{\linewidth}{0.5mm}} % Defines a new command for the horizontal lines, change thickness here

        \center % Center everything on the page

        %----------------------------------------------------------------------------------------
        %	HEADING SECTIONS
        %----------------------------------------------------------------------------------------
        \textsc{\LARGE Đại học Quốc Gia Hà Nội}\\[0.5cm]
        \textsc{\LARGE Trường đại học Khoa học tự nhiên}\\[0.5cm] % Name of your university/college
        \textsc{\LARGE Khoa Toán - Cơ - Tin học}\\[0.5cm]

        \includegraphics[scale=0.2]{HUS-logo.jpg}\\[0.5cm]

        \textsc{\Large Chuyên ngành: Khoa học dữ liệu}\\[0.5cm] % Major heading such as course name


        %----------------------------------------------------------------------------------------
        %	TITLE SECTION
        %----------------------------------------------------------------------------------------

        \HRule \\[0.4cm]
        { \huge \bfseries BÁO CÁO CUỐI KỲ}\\[0.4cm] % Title of your document
        \HRule \\[1.5cm]

        \textsc{\Large Môn học: Toán rời rạc và thuật toán}\\[1cm] % Minor heading such as course title


        \textsc{\Large Đề tài: Lắp ghép các đoạn DNA \\ sử dụng tiếp cận của lý thuyết đồ thị}\\[1cm]


        %----------------------------------------------------------------------------------------
        %	AUTHOR SECTION
        %----------------------------------------------------------------------------------------
        \begin{minipage}{0.4\textwidth}
            \begin{flushleft} \large
            \emph{Giảng viên hướng dẫn:} \\
            PGS. TS. Nguyễn Thị Hồng Minh % Supervisor's Name
            \end{flushleft}
        \end{minipage}\\[0.5cm]

        \begin{minipage}{0.4\textwidth}
        \begin{flushleft} \large
        \emph{Nhóm học viên thực hiện:}\\
        Nguyễn Chí Thanh \\
        MSHV: 21007925 \\ % Your name
        Vũ Ngọc Đại \\
        MSHV: 21007977 \\
        Vũ Minh Hưng \\
        MSHV: 21007973 \\
        Lê Diệu Thúy \\
        MSHV: 21007922 \\
        Lớp: Khoa học dữ liệu - K4
        \end{flushleft}
        \end{minipage}


        % If you don't want a supervisor, uncomment the two lines below and remove the section above
        %\Large \emph{Author:}\\
        %John \textsc{Smith}\\[3cm] % Your name

        %----------------------------------------------------------------------------------------
        %	DATE SECTION
        %----------------------------------------------------------------------------------------

        % I don't want day because it is English
        % {\large \today}\\[2cm] % Date, change the \today to a set date if you want to be precise

        %----------------------------------------------------------------------------------------
        %	LOGO SECTION
        %----------------------------------------------------------------------------------------

        %\includegraphics{logo/rsz_3logo-khtn.png}\\[1cm] % Include a department/university logo - this will require the graphicx package

        %----------------------------------------------------------------------------------------

        \vfill % Fill the rest of the page with whitespace

    \end{titlepage}


    \cleardoublepage
    \pagenumbering{gobble}
    \tableofcontents
    \newpage
    \listoffigures
    \newpage
    \glsaddall 
    \renewcommand*{\glossaryname}{Danh mục các từ viết tắt}
    \renewcommand*{\acronymname}{Danh sách từ viết tắt}
    \renewcommand*{\entryname}{Viết tắt}
    \renewcommand*{\descriptionname}{Viết đầy đủ}
    \printnoidxglossary
    \cleardoublepage
    \pagenumbering{arabic}

    %\maketitle

    \newpage

    \nocite{*}

    \begin{center}
    \section*{LỜI MỞ ĐẦU}
    \end{center}
    \addcontentsline{toc}{section}{{\bf LỜI MỞ ĐẦU}\rm}

    Được xây dựng dựa trên các công trình trước đây về giải trình tự bằng lai ghép, lắp ráp các mảnh là một phương pháp mới được khám phá để xác định xem một chuỗi DNA được lắp ráp lại có khớp với chuỗi DNA ban đầu hay không.
    Một cách cụ thể để phân tích phương pháp này là sử dụng các khái niệm từ lý thuyết đồ thị.
    Bằng cách xây dựng các mô hình dữ liệu dựa trên những ý tưởng này, có thể đưa ra nhiều kết luận khác nhau về vấn đề ban đầu liên quan đến các chuỗi DNA được lắp ráp lại.
    Trong bài báo này ta sẽ trình bày chi tiết cách tiếp cận này để lắp ráp đoạn DNA và trình bày một số chứng minh lý thuyết trong quá trình lắp ráp bao gồm cả định lý BEST.
    Mặt khác, ta sẽ khám phá bài toán đường đi Euler và vai trò của nó trong việc hỗ trợ lắp ráp các đoạn DNA, và các ứng dụng khác gần đây của lý thuyết đồ thị trong lĩnh vực tin sinh học.

    \newpage
    \section{Mở đầu}
    
    Trong những năm gần đây, các nhà khoa học và các nhà nghiên cứu đã tập trung vào giải trình tự và lắp ráp các đoạn DNA với hy vọng nâng cao khả năng để tái tạo lại toàn bộ chuỗi DNA dựa trên các mảnh dữ liệu mà họ có thể thu được.
    Phức tạp nảy sinh khi lắp ráp các đoạn nhỏ do dữ liệu không hoàn hảo.
    Các sợi thường được lặp lại nhiều lần với nhiều độ dài khác nhau.
    Kết quả là, cấu hình của bộ gen ban đầu không hề dễ dàng như việc ghép mảnh này vào mảnh ghép tiếp theo.

    Ta sẽ thảo luận về các cách tiếp cận mới được khám phá gần đây để lắp ráp các đoạn DNA sử dụng các thành phần từ lý thuyết đồ thị,
    bao gồm đồ thị de Bruijin và chu trình Euler để khắc phục các sai lệch trong quá trình lắp ráp các đoạn DNA.
    Cụ thể, cách tiếp cận này bắt nguồn từ các công trình giải trình tự bằng lai ghép, bao gồm xây dựng các đồ thị có hướng dựa trên dữ liệu DNA được cung cấp và đếm số chu trình Euler trong các đồ thị này.

    Bài báo này sẽ cung cấp một số thông tin cơ bản sơ bộ từ lý thuyết đồ thị.
    Ngoài một số định nghĩa, ta sẽ chứng minh hai định lý, định lý BEST và định lý cây ma trận, cả hai đều được sử dụng trong việc đếm số chu trình Euler trong đồ thị có hướng.

    Hơn nữa, bài báo sẽ khảo sát ngắn gọn tình hình nghiên cứu của bài toán giải trình tự và lắp ráp các đoạn DNA,
    chẳng hạn các vấn đề phát sinh và cách chúng được giải quyết bằng lý thuyết đồ thị.
    Ta sẽ xem cách các đồ thị được xây dựng sử dụng dữ liệu DNA và cách mà các đồ thị này biểu diễn bài toán.
    Vấn đề chính mà ta quan tâm là bài toán đường đi Euler trong \cite{pevzner2001eulerian}, \cite{pevzner2001new}, đây là một nỗ lực gần đây để đơn giản hóa đồ thị DNA thông qua một chuỗi các phép biến đổi trên đồ thị.
    Các phép biến đổi như vậy bao gồm các phép tách cạnh khác nhau cho các trường hợp có một và nhiều cạnh trong đồ thị Euler có hướng.

    Phần kết luận thảo luận sự phân tích của trình tự DNA với nanopores, như được trình bày chi tiết trong \cite{bokhari2005parallel} và vai trò của lý thuyết độ thị trong việc giải quyết các vấn đề nảy sinh.
    Đây là một ứng dụng gần đây hơn của lý thuyết đồ thị được đưa vào sử dụng trong lĩnh vực tinh sinh học.
    
    \section{Các định nghĩa trong lý thuyết đồ thị}

    Lý thuyết đồ thị là một lĩnh vực của toán học có nhiều ứng dụng trong thế giới khoa học.
    Ta giả định người đọc đã quen thuộc với những nền tảng cơ bản của lý thuyết đồ thị, chẳng hạn như được đề cập trong \cite{tucker2006applied} và \cite{west2001introduction}.
    Các thuật ngữ cơ bản trong bài báo theo các thuật ngữ của \cite{west2001introduction}.

    Ta có thẻ di chuyển qua các phần tử của đồ thị theo nhiều cách khác nhau.
    Một đường đi là một hành trình bất kỳ qua một đồ thị tạo ra một danh sách các đỉnh và các cạnh sao cho một cạnh kết nối các đỉnh ở hai bên của cạnh này.
    Một vết là một đường đi mà không có bất kỳ cạnh nào được lặp lại.
    Một vết khép kín là một chu trình và danh sách các đỉnh được biểu diễn theo thứ tự tuần hoàn.
    Hơn nữa, một đường dẫn là một vết mà trong đó không có đỉnh nào được lặp lại sao cho các đỉnh có thể được liệt kê theo thứ tự liên tiếp từ đỉnh này liền kề với đỉnh tiếp theo.
    Trong trường hợp đồ thị có hướng, các cạnh của một vết hoặc một chu trình phải nhất quán về hướng.

    Một vết hoặc một chu trình đi qua các cạnh trong đồ thị một lần và chính xác một lần được gọi là Euler.
    Các vết hoặc các chu trình Euler được cho phép đi qua các đỉnh trong đồ thị nhiều hơn một lần.
    Số chu trình Euler trong đồ thị có thể được tính toán với công thức từ định lý BEST, định lý này sẽ được chứng minh ở phần sau trong bài báo.

    Một định nghĩa cuối cùng từ lĩnh vực lý thuyết đồ thị mà ta sử dụng trong bài báo là đồ thị de Bruijin.
    Đồ thị de Bruijin là một đồ thị có hướng với các đỉnh biểu diễn các chuỗi ký tự từ bảng chữ cái và các cạnh cho biết vị trí mà chuỗi có thể chồng lên nhau như được định nghĩa trong \cite{de1946combinatorial}.

    Việc xây dựng đồ thị này phụ thuộc vào một tập các mảnh hoặc các đoạn có độ dài $l$ từ dãy cụ thể hiện có.
    Mỗi đỉnh được gán nhãn bởi một đoạn có chiều dài $l-1$ và cạnh có hướng tồn tại giữa hai đỉnh đại diện cho một trong đoạn có chiều dài $l$.
    Cụ thể, ký hiệu đầu tiên trong đoạn xuất phát từ đỉnh gửi cạnh và tương tự, ký tự cuối cùng biểu diễn đỉnh nhận.
    Vì vậy, các ký tự còn lại trong cả hai đỉnh bao gồm sẽ đánh dấu các cạnh.

    \begin{figure}[h!]
        \centering
        \includegraphics[width=0.4\textwidth]{1.png}
        \caption{Một đồ thị de Bruijin cho chuỗi "0110101" với các đoạn chiều dài là 3.}
        \label{fig:1}
    \end{figure}

    Ví dụ, ta xây dựng một đồ thị de Bruijin cho chuỗi "0110101" sử dụng các đoạn có độ dài $l=3$.
    Bốn bộ ba có mặt trong chuỗi là 011, 110, 101 (xuất hiện hai lần) và 010.
    Hình \ref{fig:1} cho thấy đồ thị de Bruijin trùng với chuỗi này.
    Ta cần lưu ý rằng các đỉnh đại diện cho chuỗi con đầu tiên và cuối cùng có độ dài là 2 trong mỗi bộ ba, và cạnh có hướng giữa hai đỉnh đại diện cho bộ ba tương ứng.

    \section{Giải trình tự và lắp ráp các đoạn DNA}

    Để đơn giản, ta xem giải trình tự chuỗi DNA giống như là quá trình xây dựng trò chơi xếp hình của trẻ em như được thể hiện trong \cite{pevzner2001eulerian}.
    Điểm khác biệt ở chỗ các nhà khoa học đang phải xử lý hàng trăm mảnh có kích thước khác nhau và một số mảnh giống hệt nhau.
    Các "mảnh ghép" là các chuỗi DNA khác nhau đọc được và hình được xếp hoàn chỉnh là toàn bộ chuỗi ban đầu hoặc bộ gen trông như thế nào.
    Thông tin được cung cấp về khía cạnh này của tin sinh học đến từ các phần khác nhau của công trình \cite{jones2004introduction}.

    Cụ thể, lắp ráp các đoạn DNA là một kỹ thuật trong phòng thí nghiệm xác định cấu hình củ toàn bộ bộ gen dựa trên một loạt các chuỗi được đọc từ bộ gen đó.
    Vì một chuỗi DNA đầy đủ bao gồm hàng triệu nucleotide nên không có cách nào để nắm được toàn bộ cấu trúc một cách rõ ràng.
    Với công nghệ ngày nay, các nhà khoa học và các nhà nghiên cứu đã có thể thu được các đoạn DNA có độ dài từ 500 đến 1200 nucleotide.
    Tất cả các đoạn này bao gồm bốn loại ba-zơ tạo nên DNA: Adenine (A), Thymine (T), Guanine (G) và Cystosine (C).
    Có thể có nhiều cách để tái tạo lại chuỗi ban đầu từ các mảnh DNA nhưng chỉ có một trong số đó là đúng.

    Có rất nhiều vấn đề và sự phức tạp phát sinh trong quá trình lắp ráp các đoạn khiến cho việc này khó hơn rất nhiều so với việc chỉ đơn giản là xây dựng một hình xếp.
    Trong phần miêu tả các đoạn đọc được tồn tại một tỷ lệ lỗi do máy giải trình tự tạo ra.
    Tỷ lệ sai này khoảng 1 \% đến 3 \% của DNA do máy giải trình tự tạo ra có chứa lỗi và không đại diện một cách thích hợp cho một phần của chuỗi đầy đủ.

    Một vấn đề nữa là chuỗi DNA có cấu trúc xoắn kép.
    Dựa theo công trình \cite{watson2003molecular}, ta biết rằng với mỗi chuỗi đơn DNA tồn tại một chuỗi bổ sung, A khớp với T, C khớp với G.
    Vì vây, khi nhìn vào các đoạn đọc được từ một chuỗi DNA, ta không thể xác định được liệu đoạn này đến từ chuỗi mong muốn hay đến từ chuỗi bổ sung.

    Mặc dù tỷ lệ lỗi và việc không biết được đoạn DNA đến từ chuỗi quan tâm hay chuỗi bổ sung cản trở một số bước trong quá trình giải trình tự và lắp ráp các đoạn DAN, nhưng điều ta quan tâm và dễ nhầm lẫn hơn cả đến từ các chuỗi được lặp đi lặp lại.
    Bộ gen người có một lượng lớn các chuỗi lặp đi lặp lại nhiều lần.
    Hơn nữa, nếu một chuỗi lặp lại có độ dài lớn hơn độ dài đoạn có thể đọc được thì việc xây dựng lại bộ gen gần như là không thể.
    Ví dụ, ta xét một bộ gen cụ thể có chuỗi lặp lại có độ dài 2000 nucleotide.
    Ngày nay không có một máy giải trình tự nào có thể đọc được một đoạn DNA co độ dài như vậy, vì vậy không có cách nào để đọc toàn bộ đoạn lặp lại.

    Hầu hết các nỗ lực sửa lỗi trong lắp ráp các mảnh DNA nhằm khắc phục vấn đề lặp lại của các chuỗi.
    Các cách tiếp cận trước đây thường theo thuật toán "overlap-layout-concensus".
    Bước đầu tiên liên quan đến việc khớp tất cả các đoạn đọc được và tìm các khoảng chồng lấn.
    Điều này được thực hiện bằng cách nhìn vào chuỗi bắt đầu của một đoạn đọc được và chuỗi kết thúc của một đoạn đọc được khác.
    Bước bố cục là xây dựng các chuỗi và đây là việc khó khăn nhất.
    Ta ghi nhớ các chuỗi lặp lại, một nỗ lực được thực hiện để tìm thứ tự của các đoạn đọc được dọc theo một chuỗi DNA đầy đủ và xếp chúng lại với nhau.
    Thành phần cuối cùng của thuật toán này là tìm ra cách các chuỗi sẽ xuất hiện dựa trên bố cục tạo ra ở bước trước.
    Cách tiếp cận ta sẽ thảo luận trong bài báo này bỏ qua phác thảo của quá trình ba bước ở trên và khi làm vậy, bài toán trở thành bài toán mang tính xác suất.
    
    \section{So sánh chuỗi DNA với đồ thị de Bruijin}

    Giải trình tự bằng lai ghép là một kỹ thuật trong phòng thí nghiệm gần giống như nhìn vào các đoạn của DNA,
    được gọi là mảng DNA, nhưng không nên nhầm lẫn với lắp ráp các đoạn DNA, vì chúng khác nhau.
    Cụ thể, giải trình tự bằng lai ghép (SBD) chủ yêu dựa vào sự liên kết của một đoạn DNA mục tiêu chưa biết với một dãy các đoạn tổng hợp ngắn hơn gọi là đầu dò.
    Các đầu dò này có dộ dài khoảng từ 8 đến 30 nucleotide và chúng liên kết với mục tiêu dựa trên phần bổ sung Watson-Crick được đề cập trong \cite{de1946combinatorial}.

    Tuy nhiên, chính cách tiếp cận lý thuyết đồ thị sử dụng SBH \cite{idury1995new} đã đề xuất cách tiếp cận tương tự trong lĩnh vực lắp ráp các đoạn DNA.
    Trong bài báo của mình, các tác giả đã định nghĩa các quy tắc để xây dựng một đồ thị có hướng dựa trên các đoạn DNA.
    Cụ thể hơn, các quy tắc này là để xây dựng một đồ thị de Bruijin dựa trên các đoạn DNA có độ dài $k$.
    Đồ thị có các định được gán nhãn bằng một đoạn có độ dài là $k-1$, các cạnh được gán nhãn bằng đoạn có độ dài $k$, kết nối hai đỉnh kề trong chuỗi.

    \begin{figure}[h!]
        \centering
        \includegraphics[width=0.6\textwidth]{2.png}
        \caption{Một đồ thị de Bruijin cho chuỗi ATGTGCCGCA.}
        \label{fig:2}
    \end{figure}

    Hình \ref{fig:2} min họa một đồ thị de Bruijin cho một chuỗi đơn giản ATGTGCCGCA, chuỗi này đã được sử dụng bởi \cite{idury1995new}.
    Ở đây, các đoạn DNA đọc được có độ dài là 3.
    Vì vậy, các đỉnh được đánh dấu bằng các đoạn có độ dài là 2 và các cạnh được đánh dấu bằng các đoạn có độ dài là 3, biểu diễn cho các bộ ba thu được từ chuỗi ban đầu.
    Ta cần chú ý rằng trong đồ thị trên chỉ có duy nhất một vết Euler vì đây chỉ là một trường hợp đơn giản.
    Nhiều công trình \cite{pevzner2000computational}, \cite{pevzner1995dna} đã phát triển thêm ý tưởng này.

    \begin{figure}[h!]
        \centering
        \includegraphics[width=0.6\textwidth]{3.png}
        \caption{Một đồ thị được xây dựng từ tập các đoạn đọc được $S$}
        \label{fig:3}
    \end{figure}

    Bây giờ, ta sẽ xây dựng một đồ thị de Bruijin khác bằng cách sử dụng các quy tắc tương tự như trên.
    Với tập hợp các đoạn $S = \lbrace ATG, TGG, TGC, GTG, GGC, GCA, GCG, CGT \rbrace$,
    đồ thị sẽ có các đỉnh được gán nhãn bởi các đoạn có độ dài là 2 và các cạnh được gán nhãn bởi các đoạn có độ dài là 3.
    Tuy nhiên, đồ thị được tạo ra cho tập hợp các đoạn đọc được này phức tạp hơn đồ thị trước đó.
    Cụ thể, có hai vết khép kín khác nhau cho toàn bộ đồ thị, tạo ra các chuỗi ATGGCGTGCA và ATGCGTGGCA.
    Ở hình \ref{fig:3}, ta lưu ý rằng cạnh nét được phụ được thêm vào để kết nối đỉnh đầu tiên và đỉnh cuối cùng AT và CA tạo thành một đồ thị có hướng Euler.

    Như đã được đề cập ở trên, phương pháp tiếp cận lắp ráp các đoạn DNA này được một số nà nghiên cứu gọi là phương pháp đường đi Euler, tạo ra một bài toán về xác suất.
    Với ví dụ trên, ta thấy với tập hợp mảnh đã cho ta có thể xây dựng hai chuỗi khác nhau dựa trên việc có hai chu trình Euler khác nhau.
    Vì vậy, xác suất là 1/2 ta sẽ thu được bộ gen đúng với thực tế tùy thuộc vào chu trình Euler mà ta tìm ra.
    Vì ta đã thêm một cạnh ảo bổ sung làm cho đồ thị khép kín, nên ta có thể đếm số chu trình Euler thay vì vết.

    Tổng quát hơn, khi ta được cung cấp dữ liệu đầy đủ và không có lỗi, ta có thể thấy rằng xác suất tái tạo lại chuỗi DNA ban đầu với một tập hợp các đoạn là:

    \begin{equation}
        \dfrac{1}{\text{tổng số chu trình/đường đi Euler trong đồ thị de Bruijin}}
    \end{equation}

    Cùng với đó, có một định lý được dùng để đếm số chu trình Euler trong đồ thị có hướng, được gọi là định lý BEST và ta sẽ chứng minh, tiếp theo là một định lý liên quan khác và chứng minh của định lý này.

    \section{Định lý BEST}

    Tồn tại một công thức tính số chu trình Euler trong đồ thị có hướng.
    Được đặt tên theo những người tìm ra là de Bruijin, van Aardenne-Ehrenfest, Smith và Tutte \cite{van1951circuits}, \cite{tutte1941unicursal}.
    Định lý BEST được phát biểu như sau:

    \begin{dl}
        Cho một đồ thị có hướng liên thông $G$ và tập các đỉnh $V(G)=\lbrace v_1, v_2, \dots, v_n \rbrace$ tất cả đều có bậc chắn, số chu trình Euler là $\lvert s(G) \rvert$ được tính theo công thức sau, với $\lvert t_i (G) \rvert$ là số cây khung hướng đến gốc là một đỉnh $v_i$ bất kỳ trong $G$:

        \begin{equation*}
            \lvert s(G) \rvert = \lvert t_i (G) \rvert \prod_{j=1}^n \big( d^+ (v_j) - 1 \big)!
        \end{equation*}
    \end{dl}

    Chứng minh của định lý này được trình bày trong \cite{bollobas1998graduate}.

    \textbf{Chứng minh:}
    Ta được cho một đa đồ thị có hướng $G$ với tập đỉnh $V(G)=\lbrace v_1, v_2, \dots, v_n \rbrace$ và bậc ra và bậc vào của một đỉnh $v_i$ bằng nhau, ký hiệu là $d^+ (v_i) = d^- (v_i)$.
    Nếu điều kiện trên thỏa mãn, ta biết rằng đồ thị $G$ có ít nhất một chu trình có hướng Euler theo một định lý trong \cite{bollobas1998graduate}.

    Mặt khác, ta đặt $E$ là tập của các chu trình Euler có hướng và $E_i$ là tập các vết Euler có hướng bắt đầu và kết thúc với đỉnh $v_i$.
    Số lần mà từng chu trình Euler đi qua đỉnh $v_i$ tương đương với bậc vào (hoặc bậc ra) của $v_i$, vì vậy số vết Euler bắt đầu với đỉnh $v_i$ có thể được tính theo công thức:

    \begin{equation}
        \lvert E_i \rvert = d^+ (v_i) \lvert E \rvert = d^- (v_i) \lvert E \rvert
    \end{equation}

    Công thức trên cho biết số vết Euler bắt đầu và kết thúc với $v_i$ tương đương với bậc vào hoặc bậc ra của $v_i$ nhân với tổng số chu trình Euler trong đồ thị $G$.

    Ta đặt $T_i$ là tập các cây khung hướng đến $v_i$ và sử dụng tập này trong hàm ánh xạ $f_i: E_i \rightarrow T_i$.
    Ánh xạ này có đầu vào là tập các vết Euler bất đầu và kết thúc với $v_i$ trong đồ thị $G$ và đầu ra là tập các cây khung hướng đến $v_i$ trong đồ thị $G$.
    Để các ký hiệu đơn giản trong chứng minh, ta sẽ sử dụng $i=1$ nên $f_1: E_1 \rightarrow T_1$.

    Cho $S$ là một vết Euler từ tập $E_1$, ánh xạ này xây dựng một tập các cạnh $e_j$ theo cách sau.
    Cạnh đầu tiên trong $S$ thoát ra khỏi $v_1$ không được vẽ, nghĩa là $j=2,\dots, n$ cho mỗi cạnh $e_j$.
    Hơn nữa, mỗi cạnh $e_j$ xuất hiện khi $S$ đi qua một đình $v_j$ lần cuối và không quay lại đỉnh này lần nữa trên vết Euler.
    Kết quả từ ánh xạ này là một cây có hướng $T$ với các đỉnh $v_1, \dots, v_n$ và tập $e_2, \dots, e_n$ nằm trong tập các cây khung hướng đến đỉnh $v_1$.
    Hình \ref{fig:3} cho thấy một ví dụ của ánh xạ dựa trên chu trình Euler xác định trong $G$.

    Ta sẽ chứng minh $T \in T_1$, ta phải chứng minh hai điều:

    \begin{enumerate}[label=(\alph*)]
        \item $T$ thực sự là một cây
        \item $T$ hướng đến đỉnh $v_1$.
    \end{enumerate}

    \newpage
    \printbibliography[title={TÀI LIỆU THAM KHẢO}]

\end{document}